\documentclass[11pt]{beamer}
\usetheme{Madrid}
\setbeamertemplate{navigation symbols}{}
\usecolortheme{default}
\usepackage[utf8]{inputenc}
\usepackage[ngerman]{babel}
\usepackage{amsmath, amssymb}
\usepackage{algorithm2e}
\usepackage{graphicx}
\usepackage{tikz}


%%%% TITEL-INFOS %%%%
\title{QuickHull und Gaußsche Trapezformel}
\subtitle{GINF Ausarbeitung}
\author{Oliver Hahn}
\date{}

\begin{document}

%%%% TITELFOLIE %%%%
\begin{frame}
  \titlepage
\end{frame}


%%%% INHALTSVERZEICHNIS %%%%
\begin{frame}{Übersicht}
  \tableofcontents
\end{frame}

%%%% SECTION: QUICKHULL %%%%
\section{QuickHull}

%%%% FOLIE: PROBLEM %%%%
\begin{frame}{QuickHull: Problem und Aufgabe}
  \textbf{Gegeben:} Endliche Punktmenge $S = \{p_1, \dots, p_n\}$ in der Ebene
  
  \vspace{0.5cm}
  
  \textbf{Gesucht:} Konvexe Hülle $CH(S)$
  
  \vspace{0.5cm}
  
  \textbf{Anschaulich:}
  \begin{itemize}
    \item Kleinstes Polygon, das alle Punkte enthält
    \item Wie ein Gummiband um die Punkte
    \item Rand als Polygonzug in zyklischer Reihenfolge
  \end{itemize}
  
  \vspace{0.3cm}
  
  \begin{center}
    \textit{Details in der Ausarbeitung, Abschnitt 1.1}
  \end{center}
\end{frame}

%%%% FOLIE: ALGORITHMISCHE IDEE %%%%
\begin{frame}{QuickHull: Algorithmische Idee}
  \textbf{Divide-and-Conquer-Ansatz} (ähnlich QuickSort)
  
  \vspace{0.5cm}
  
  \begin{enumerate}
    \item Bestimme Punkte mit min/max x-Koordinate: $p_{min}$, $p_{max}$
    \item Diese liegen garantiert auf der konvexen Hülle
    \item Gerade durch $p_{min}$ und $p_{max}$ teilt Punktmenge in zwei Halbebenen
    \item Für jede Halbebene: Finde Punkt mit größtem Abstand zur Geraden
    \item Rekursive Zerlegung in Teilmengen
    \item Terminierung: Keine Punkte mehr außerhalb der Strecken
  \end{enumerate}
  
  \vspace{0.3cm}
  
  \begin{center}
    \textit{Details in der Ausarbeitung, Abschnitt 1.2}
  \end{center}
\end{frame}

%%%% FOLIE: KREUZPRODUKT %%%%
\begin{frame}{QuickHull: Kreuzprodukt als Werkzeug}
  \textbf{Zentrales Werkzeug:} Kreuzprodukt
  
  \vspace{0.3cm}
  
  \begin{equation*}
    \text{cross}(a, b, c) = (b_x - a_x)(c_y - a_y) - (b_y - a_y)(c_x - a_x)
  \end{equation*}
  
  \vspace{0.3cm}
  
  \textbf{Vorzeichen:}
  \begin{itemize}
    \item Positiv: $c$ liegt \textbf{links} von $a \to b$
    \item Negativ: $c$ liegt \textbf{rechts} von $a \to b$
    \item Null: Punkte sind kollinear
  \end{itemize}
  
  \vspace{0.3cm}
  
  \textbf{Betrag:}
  \begin{itemize}
    \item Proportional zur Fläche des Dreiecks $abc$
    \item Proportional zum Abstand von $c$ zur Geraden $ab$
  \end{itemize}
  
  \vspace{0.3cm}
  
  \begin{center}
    \textit{Details in der Ausarbeitung, Abschnitt 1.3}
  \end{center}
\end{frame}

%%%% FOLIE: FUNKTIONSWEISE %%%%
\begin{frame}{QuickHull: Funktionsweise}
  Für gerichtete Kante $a \to b$:
  
  \vspace{0.3cm}
  
  \begin{itemize}
    \item Betrachte alle Punkte links der Kante:
    \begin{equation*}
      L_{ab} = \{p \in S \mid \text{cross}(a,b,p) > 0\}
    \end{equation*}
    
    \item Falls $L_{ab} = \emptyset$: Kante ist Teil der Hülle
    
    \item Sonst: Wähle Punkt mit maximalem Abstand:
    \begin{equation*}
      P = \operatorname*{arg\,max}_{p \in L_{ab}} \text{cross}(a,b,p)
    \end{equation*}
    
    \item $P$ ist Hüllpunkt
    
    \item Rekursion auf $L_{aP}$ und $L_{Pb}$
  \end{itemize}
  
  \vspace{0.3cm}
  
  \begin{center}
    \textit{Details und Beweis in der Ausarbeitung, Abschnitt 1.3}
  \end{center}
\end{frame}

%%%% FOLIE: KOMPLEXITÄT %%%%
\begin{frame}{QuickHull: Komplexität}
  \textbf{Laufzeit:}
  
  \vspace{0.5cm}
  
  \begin{itemize}
    \item \textbf{Durchschnitt:} $\mathcal{O}(n \log n)$
    \vspace{0.3cm}
    \item \textbf{Worst Case:} $\mathcal{O}(n^2)$
    \begin{itemize}
      \item Bei speziellen Punktverteilungen
      \item Wenn in jedem Schritt nur ein Punkt eliminiert wird
    \end{itemize}
  \end{itemize}
  
  \vspace{0.5cm}
  
  \textbf{Eigenschaften:}
  \begin{itemize}
    \item Terminierung garantiert
    \item Jeder Rekursionsschritt identifiziert mindestens einen Hüllpunkt
    \item Innere Punkte werden systematisch eliminiert
  \end{itemize}
  
  \vspace{0.3cm}
  
  \begin{center}
    \textit{Details in der Ausarbeitung, Abschnitt 1.3}
  \end{center}
\end{frame}

%%%% SECTION: GAUSS TRAPEZFORMEL %%%%
\section{Gaußsche Trapezformel}

%%%% FOLIE: PROBLEM TRAPEZ %%%%
\begin{frame}{Gaußsche Trapezformel: Problem}
  \textbf{Gegeben:} Einfaches Polygon mit $n$ Eckpunkten
  \begin{equation*}
    p_0, p_1, \dots, p_{n-1}, \quad p_i = (x_i, y_i) \in \mathbb{R}^2
  \end{equation*}
  
  \vspace{0.5cm}
  
  \textbf{Gesucht:} Flächeninhalt $A$ des Polygons
  
  \vspace{0.5cm}
  
  \textbf{Voraussetzung:}
  \begin{itemize}
    \item Polygon ist einfach (keine Selbstschneidungen)
    \item Punkte in konsistenter zyklischer Reihenfolge
  \end{itemize}
  
  \vspace{0.3cm}
  
  \begin{center}
    \textit{Details in der Ausarbeitung, Abschnitt 2.1}
  \end{center}
\end{frame}

%%%% FOLIE: FORMEL %%%%
\begin{frame}{Gaußsche Trapezformel: Die Formel}
  \textbf{Auch bekannt als:} Shoelace-Formel
  
  \vspace{0.5cm}
  
  \textbf{Signierte Fläche:}
  \begin{equation*}
    A_{\text{signed}} = \frac{1}{2} \sum_{i=0}^{n-1} (x_i y_{i+1} - x_{i+1} y_i)
  \end{equation*}
  
  wobei $p_n = p_0$ (zyklisch)
  
  \vspace{0.5cm}
  
  \textbf{Flächeninhalt:}
  \begin{equation*}
    A = |A_{\text{signed}}|
  \end{equation*}
  
  \vspace{0.5cm}
  
  \textbf{Vorzeichen:}
  \begin{itemize}
    \item Positiv: Counter-Clockwise (CCW)
    \item Negativ: Clockwise (CW)
  \end{itemize}
  
  \vspace{0.3cm}
  
  \begin{center}
    \textit{Details in der Ausarbeitung, Abschnitt 2.2--2.3}
  \end{center}
\end{frame}

%%%% FOLIE: FUNKTIONSWEISE TRAPEZ %%%%
\begin{frame}{Gaußsche Trapezformel: Funktionsweise}
  \textbf{Grundidee:}
  
  \vspace{0.3cm}
  
  \begin{itemize}
    \item Jede Kante $p_i \to p_{i+1}$ trägt einen vorzeichenbehafteten Trapezanteil bei
    \item Kreuzproduktterm für eine Kante:
    \begin{equation*}
      c_i = x_i y_{i+1} - x_{i+1} y_i
    \end{equation*}
    
    \item Entspricht doppelter signierter Fläche des Dreiecks vom Ursprung
    
    \item Positive und negative Anteile heben sich für innere Bereiche auf
    
    \item Am Ende verbleibt genau die Polygonfläche
  \end{itemize}
  
  \vspace{0.5cm}
  
  \textbf{Keine Integration notwendig!}
  
  \vspace{0.3cm}
  
  \begin{center}
    \textit{Details in der Ausarbeitung, Abschnitt 2.3}
  \end{center}
\end{frame}

%%%% FOLIE: ANWENDUNG AUF HULL %%%%
\begin{frame}{Anwendung auf die konvexe Hülle}
  \textbf{Problem:} QuickHull liefert Punkte nicht in zyklischer Reihenfolge
  
  \vspace{0.5cm}
  
  \textbf{Lösung:} Sortierung nach Polarwinkel
  
  \vspace{0.3cm}
  
  \begin{enumerate}
    \item Berechne Schwerpunkt (Zentroid):
    \begin{equation*}
      c = \frac{1}{n} \sum_{i=0}^{n-1} p_i
    \end{equation*}
    
    \item Sortiere Punkte nach Polarwinkel bezüglich $c$:
    \begin{equation*}
      \theta_i = \operatorname{atan2}(y_i - c_y, x_i - c_x)
    \end{equation*}
    
    \item Konvexe Hülle ist sternförmig $\Rightarrow$ einfaches Polygon garantiert
  \end{enumerate}
  
  \vspace{0.3cm}
  
  \begin{center}
    \textit{Details in der Ausarbeitung, Abschnitt 2.4}
  \end{center}
\end{frame}

%%%% FOLIE: KOMPLEXITÄT TRAPEZ %%%%
\begin{frame}{Gaußsche Trapezformel: Komplexität}
  \textbf{Zeitkomplexität:}
  
  \vspace{0.5cm}
  
  \begin{itemize}
    \item Berechnung des Zentroids: $\mathcal{O}(n)$
    \vspace{0.3cm}
    \item Shoelace-Summe: $\mathcal{O}(n)$
    \vspace{0.3cm}
    \item Sortierung nach Polarwinkel: $\mathcal{O}(n \log n)$
  \end{itemize}
  
  \vspace{0.5cm}
  
  \textbf{Gesamtkomplexität:} $\mathcal{O}(n \log n)$
  
  \vspace{0.3cm}
  
  (Sortierung dominiert)
  
  \vspace{0.5cm}
  
  \begin{center}
    \textit{Details in der Ausarbeitung, Abschnitt 2.5}
  \end{center}
\end{frame}

%%%% SECTION: ZUSAMMENFASSUNG %%%%
\section{Zusammenfassung}

%%%% FOLIE: ZUSAMMENFASSUNG %%%%
\begin{frame}{Zusammenfassung}
  \textbf{QuickHull:}
  \begin{itemize}
    \item Divide-and-Conquer-Algorithmus für konvexe Hülle
    \item Kreuzprodukt zur Orientierung und Abstandsberechnung
    \item Durchschnittliche Laufzeit: $\mathcal{O}(n \log n)$
  \end{itemize}
  
  \vspace{0.5cm}
  
  \textbf{Gaußsche Trapezformel:}
  \begin{itemize}
    \item Effiziente Flächenberechnung für einfache Polygone
    \item Direkte Berechnung aus Koordinaten
    \item Laufzeit: $\mathcal{O}(n \log n)$ (mit Sortierung)
  \end{itemize}
  
  \vspace{0.5cm}
  
  \textbf{Kombination:}
  \begin{itemize}
    \item QuickHull berechnet konvexe Hülle
    \item Trapezformel berechnet deren Fläche
  \end{itemize}
\end{frame}

%%%% FOLIE: IMPLEMENTATION %%%%
\begin{frame}{Implementation}
  \begin{center}
    \Large
    \textbf{Implementation verfügbar}
    
    \vspace{1cm}
    
    \normalsize
    Eine vollständige Implementation beider Algorithmen wurde erstellt.
    
    \vspace{0.5cm}
    
    Die Implementation umfasst:
    \begin{itemize}
      \item QuickHull-Algorithmus
      \item Gaußsche Trapezformel
      \item Visualisierung der Ergebnisse
      \item Testfälle und Beispiele
    \end{itemize}
    
    \vspace{1cm}
    
    \textit{Siehe separate Implementationsdateien}
  \end{center}
\end{frame}

\end{document}